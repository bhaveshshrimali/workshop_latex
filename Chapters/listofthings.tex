\ProvidesPackage{commands}
\documentclass[11pt]{report}
\usepackage{epstopdf}
\usepackage{amsmath}
\usepackage{epsf}
\usepackage{amsfonts}
\usepackage{amssymb}
\usepackage{color}
\usepackage{mathtools}
\usepackage{placeins}
\usepackage{booktabs}
\usepackage{enumitem}
\usepackage{caption}
\usepackage[margin=0.7in, paperwidth=8.5in, paperheight=11in]{geometry}
\usepackage{amsfonts}
\usepackage{amsbsy}
\usepackage{authblk}
\usepackage{listings}
\usepackage{array}
\usepackage{titlesec}
\usepackage{bm}
\usepackage{titlesec}
\usepackage{empheq}
\usepackage[latin1]{inputenc}
\usepackage{mathtools}
\usepackage{graphicx}
\usepackage{caption}
\usepackage{subcaption}
\usepackage{multicol}
\usepackage{xspace}
\usepackage{hyperref}
\usepackage{url}
\usepackage{natbib}


\newcommand{\del}[2]{\frac{\partial {#1}}{\partial {#2}}}
\newcommand{\D}[2]{\frac{D^{\overline{\alpha}}}{\overline{\alpha !}}{#1}(#2,#2)\ {\bf x}^{\overline{\alpha}}}
\newcommand{\dv}[3]{\frac{{\rm d}^{#1}{#2}}{d{#3}^{#1}}}
\newcommand{\ddel}[5]{\frac{\partial^{ {#1} + {#2}} {#3}}{\partial {#4}^{#1} \partial{#5}^{#2}}}
\newcommand{\dev}{{\rm {\bf dev}}}
\newcommand{\proj}[1]{\frac{1}{R^2}{\bf X}\otimes{\bf X}}
\newcommand{\Ie}[1]{I^{\rm e}_{#1}}
\newcommand{\Ce}[1]{\bf C^{\rm e^{#1}}}
\newcommand{\Fe}[2]{F^{\rm e^{#2}}_{#1}}
\newcommand{\Fv}[2]{F^{\rm v^{#2}}_{#1}}
\newcommand{\C}[2]{C^{\rm {#2}}_{#1}}
\newcommand{\f}[2]{f^{\rm {#2}}_{#1}}
\newcommand{\B}[2]{B^{\rm {#2}}_{#1}}
\newcommand{\E}[2]{E^{\rm {#2}}_{#1}}
\newcommand{\fv}[2]{f^{\rm v^{#2}}_{#1}}
\newcommand{\dfv}[2]{\dot{f}^{\rm v^{#2}}_{#1}}
\newcommand{\tGam}[2]{\tilde{\Gamma}^{\rm v^{#2}}_{#1}}
\newcommand{\Gam}[2]{\Gamma^{\rm v^{#2}}_{#1}}
\newcommand{\A}[1]{\mathcal{A}_{#1}}
\newcommand{\F}[2]{F^{\rm #2}_{#1}}
\newcommand{\hpeq}{\hat{\psi}^{\rm Eq}}
\newcommand{\hpneq}{\hat{\psi}^{\rm NEq}}
\newcommand{\etak}{\eta_K({I_1,I_2,J},{\bf C^{\rm e}, B^{\rm v}})}
\newcommand{\nuk}{\nu_K({I_1,I_2,J},{\bf C^{\rm e}, B^{\rm v}})}
\newcommand{\thetak}{\theta_K({I_1,I_2,J},{\bf C^{\rm e}, B^{\rm v}})}
\newcommand{\etaj}{\eta_J({I_1,I_2,J},{\bf C^{\rm e}, B^{\rm v}})}
\newcommand{\dFv}[2]{\dot{F}^{\rm v^{#2}}_{#1}}
\newcommand{\hatpsi}{\widehat{\psi}(I_1, I_2,I^{\rm e}_1,I^{\rm e}_2,J)}
\newcommand{\hpsi}{\widehat{\psi}(I_1,I^{\rm e}_1,J)}
\newcommand{\Fh}[1]{\widehat{\mathcal{F}}\left({\bf F, \Fv{}{}}, {#1}\right)}
\newcommand{\Fhstar}[1]{\widehat{\mathcal{F}}^*\left({\bf F, \Fv{}{}}, {#1}\right)}
\newcommand{\sbar}{\overline{\bm{\sigma}}}
\newcommand{\hpsicomp}[1]{\sum_{r=1}^{2}\left\{\frac{3^{1-\alpha_r}}{2\alpha_r}\mu_r(I^{\alpha_r}_1-3^{\alpha_r})
+\frac{3^{1-a_r}}{2a_r}m_r({\Ie{1}}^{^{a_r}}-3^{a_r})\right\}
+\mu{#1}+\kappa{#1}^2}
\newcommand{\Ni}[1]{N^{(e)}_i(#1)}
\newcommand{\hNi}[1]{\hat{{N}}^{(e)}_i(#1)}
\newcommand{\Ld}{L^{\dagger}}
\newcommand{\intinfinf}{\int_{-\infty}^{\infty} \int_{-\infty}^{\infty}}
\newcommand{\LLnorm}[1]{\left\lVert{#1}\right\rVert_2}
\newcommand{\Linorm}[1]{{\left\lVert{#1}\right\rVert_\infty}}
\newcommand{\tr}{\rm tr}
\newcommand{\deldel}[2]{\frac{\partial^2 {#1}}{\partial {#2}^2}}
\newcommand{\kd}[1]{\delta_{#1}}
\newcommand{\Fie}[1]{{\bf F}^{#1}}
\newcommand{\Comp}{\emph{CompStrainStress\_Cee570.m}}
\newcommand{\Comps}{\emph{CompStrainStress\_Elem\_Cee570.m}}
\newcommand{\Feap}{\emph{FEA\_Program.m}}
\newcommand{\Elast}{\emph{Elast2d\_Elem.m}}
\newcommand{\Assem}{\em{AssemStifForc.m}}
\newcommand{\Fb}{\em{F\_bar\_int}}
\newcommand{\FormFE}{\em{FormFE.m}}
\newcommand{\Sol}{\em{SolveFE.m}}
\newcommand{\inpt}{\em{triangtwo.m}}
\newcommand{\dis}[2]{d^{{#2}}_{#1}}
\newcommand{\vel}[2]{v^{{#2}}_{#1}}
\newcommand\myeq{\stackrel{\mathclap{\normalfont\mbox{def}}}{=}}
\newcommand{\latex}{\LaTeX\xspace}
\newcommand{\tex}{\TeX\xspace}
\newcommand{\vsp}[1]{\\[#1pt]}
\newcommand{\tet}[1]{\texttt{#1}}
\newcommand{\bs}[1]{\boldsymbol{#1}}
\newcommand{\bibtex}{{\sc Bib}\tex}
\newcommand{\biblatex}{{\sc Bib}\latex}
% Matrix Spacing: 
\makeatletter
\renewcommand*\env@matrix[1][\arraystretch]{%
  \edef\arraystretch{#1}%
  \hskip -\arraycolsep
  \let\@ifnextchar\new@ifnextchar
  \array{*\c@MaxMatrixCols c}}
\makeatother

\titlespacing\section{2pt}{12pt plus 4pt minus 2pt}{6pt plus 2pt minus 2pt}
\titlespacing\subsection{2pt}{12pt plus 4pt minus 2pt}{6pt plus 2pt minus 2pt}
\titlespacing\subsubsection{2pt}{12pt plus 4pt minus 2pt}{6pt plus 2pt minus 2pt}
\titlespacing*{\title}{-2ex}{*-2ex}{-2ex}
\usepackage{color} %red, green, blue, yellow, cyan, magenta, black, white
\definecolor{mygreen}{RGB}{28,172,0} % color values Red, Green, Blue
\definecolor{mylilas}{RGB}{170,55,241}
\setlength\parindent{0pt}
\graphicspath{{Figures/}}

\title{List of Things : Sequence}
\begin{document}
\maketitle
\begin{itemize}
\item Introduction: 
\begin{itemize}
\item Work with their download issues (max 10 min !!)
\item Introduce \latex: Environment, Founder (Donald Knuth), Originally Developed in (Web) Pascal (now obsolete!). 
\item Get a survey of how everyone is doing! Did anyone experiment ? How was it ? How many know preliminary \latex. Strike a conversation. 
\item Show them something. Maybe PDE Midterm. (Try that in Word!! :P). 
\item Version Control. Give them the link to download. 
\item Download the Repository. 
\end{itemize}
\item Brief them about the environment. Viewport. Pdf Viewer.(Embedded). Do not delete any log/aux files, these are important in compiling the document for later runs. 
\item Tell them about the first thing. How to start the document ? Class ? Fontsize ? Margins. Use the package \tet{geometry}. See the time limit ? 20 mins ? 
\item Preamble: 
\begin{itemize}
\item Whatever lies between the class and \tet{begin document}. List of packages. (brief explanation of what they are meant to do. )
\item FIXME: Take a copy of the library of packages that you are using. (Early Morning)
\end{itemize}
\item Section 
\begin{itemize}
\item Explain about \tet{section*} and \tet{subsection* }
\item Explain about paragraphs and their role in writing a paper. Briefly talk about bold, undelrine, italics, \tet{text} \textit{textit} \texttt{texttt}
\end{itemize}
\item Explain difference between url and hyperlink. 
\item Graphics: We are focussing on floats. Reason being: Time constraint.
\begin{itemize}
\item Tables: 
\item Figures/Subfigures: Positioning and details. And work out examples with them. 
\item Ask them to download images, inside and outside the parent directory and deal with that! Do not forget to add \tet{graphicspath}. (Before that do a 5 minute exercise on paragraphs, sections, subsections. Adding newlines. How to add new lines. )
\end{itemize}
\item Tables: Write tables. What all details are needed ? What is the best way to add tables if desired. (Hand-input almost never!), how to do then ? Something called Macros. Range from simple to complex ! (We will try to cover the most basic ones.)
\item Make them install excel2\latex and see if they are able to correctly install it on their respective machines. 
\item Begin Math Mode ! 
\begin{itemize}
\item Short Introduction: Inline Math, Out of line math, the difference! 
\item Types of Brackets. Curls, Square Brackets, Parenthesis, super(sub)script, trigonometric functions, fractions, derivatives, Integrals (definite and indefinite)$\oint$, $\oiint$, (Install the \tet{esint} package). Summation, Dot Product, $\left<\right>$, talk about \tet{bf cdot otimes tilde hat bar dot ddot  }, \'{o}, \"{o}
\end{itemize}
\item Make them type equations, long equations, without macros and with macros. What happens when they run into multiple lines. 
\item Matrix, Macros (Again)
\item References and BibTeX $\exists$
\end{itemize}
\end{document}